\documentclass[a4paper]{scrartcl}
\usepackage[sexy]{evan}
% evan.sty can be found here: https://github.com/vEnhance/dotfiles/blob/main/texmf/tex/latex/evan/evan.sty
\title{Glossary}
\author{Mckinley Xie}

\newcommand{\term}[2]{\\$#1$  ---  #2}

\begin{document}
\maketitle
\begin{abstract}
	This is intended to be a list of basic definitions for common things in math. This will likely be updated as the year goes on, the most recent version will be at \url{https://MckinleyX.github.io/files/glossary.pdf}.

	If you think something should be added to the list, or I've made an error, contact me!

	Discord: \texttt{faefeyfa\#4843}

	Email: \mailto{mckinleyxie@gmail.com}
\end{abstract}
\section{Common symbols}
Here's a list of some of the more common symbols you'll see:
	\term{\forall}{for all} 
	\term{\exists}{there exists} 
	\term{\in}{is an element of}
	\term{\because}{because}
	\term{\therefore}{therefore}
	\term{\mathbb{Z}}{the set of all integers}
	\term{\mathbb{Z}^+}{the set of all positive integers}
	\term{\mathbb{Z}^*}{the set of all nonnegative integers}
	\term{\mathbb{R}}{the set of all real numbers}
	\term{a \mid b}{$a$ divides $b$}
	\term{\qed}{Used to denote the end of a proof. There are a \href{https://mathwithbaddrawings.com/2019/10/02/how-to-end-a-proof/}{\emph{lot}} of ways to do this, but this is what I use.}
	\term{\text{QED}}{see$\qed$}
	\term{\implies}{implies. $p \implies q$ if $q$ is true whenever $p$ is true. (Note that if $p$ is false $q$ is not necessarily false.)}
\section{Less common symbols}
Here are some symbols that are less common:
	\term{\mathbb{N}}{the set of all natural numbers -- be careful around this since not everyone agrees whether 0 is included. In IB it is.}
	\term{\mathbb{Q}}{the set of all rational numbers}
	\term{\mathbb{C}}{the set of all complex numbers}
	\term{\binom{n}{r}}{$n$ choose $r$}
	\term{\iff}{if and only if, commonly abbreviated as ``iff''. $p \iff q$ means that both $p \implies q$ and $q \implies p$.}

	\section{Sample proof}
	By request, here's a sample proof of a problem:
	\begin{problem*}
		Prove that $\frac{a + b}{2} \geq \sqrt{ab}$ for $a,b \geq 0$.
	\end{problem*}
	\begin{proof}
		Let $x=\sqrt{a}$ and $y=\sqrt{b}$. \\
		Note that 
		\[(x-y)^2 \geq 0\]
		Now, after a bit of manipulation,
		\begin{align*}
			x^2 - 2xy + y^2 &\geq 0 \\
			x^2 + y^2 &\geq 2xy \\
			\frac{x^2+y^2}{2} &\geq xy
		\end{align*}
		Finally, substituting in, we have 
		\[\frac{a+b}{2} \geq \sqrt{ab}\]
		And we are done.
	\end{proof}

\end{document}
