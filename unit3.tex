\documentclass[a4paper,12pt]{scrartcl}
\usepackage[sexy]{evan}
\usepackage{mckinley}
\usepackage{asymptote}
\usepackage[backend=biber,style=alphabetic]{biblatex}
\usepackage{xcolor}
\usepackage{amsmath}
\addbibresource{ref.bib}
\addbibresource{aux.bib}
\DeclarePairedDelimiter{\ceil}{\lceil}{\rceil}
\newcommand{\red}{\color{red}}
\newcommand{\blue}{\color{blue}}
\newcommand{\black}{\color{black}}
\usepackage{float}
\title{Unit 3 \LaTeX{} Project}
\author{Mckinley Xie}
\setlength{\marginparwidth}{2cm}
\emergencystretch=1em
\begin{document}
\maketitle
\begin{example*}
	[USAMTS 4/1/33]
	Let $m,n,k$ be positive integers such that $k \leq mn$.
	Let $S$ be the set consisting of a $(m+1)$-by-$(n+1)$
	rectangular array of points on the Cartesian plane with coordinates
	$(i,j)$ where $i,j$ are integers satisfying $0 \leq i \leq m$ and
	$0 \leq j \leq n$. The diagram below shows the example
	where $m=3$ and $n=5$, with the points of $S$ indicated by black dots:
	\begin{figure}[H]
\begin{center}
\begin{asy}
	unitsize(0.5cm);
	for(int ii = -1; ii <= 6; ++ii)
	{
		draw((-1.5,ii)--(6.5,ii),grey);
		draw((ii,-1.5)--(ii,6.5),grey);
	}
	for(int ii = 0; ii <= 3; ++ii)
	{
		for(int jj = 0; jj <= 5; ++jj)
		{
			dot((ii,jj));
		}
	}
	draw((-2,0)--(7,0),Arrow);
	draw((0,-2)--(0,7),Arrow);

	label("$x$",(7,0),E);
	label("$y$",(0,7),N);
\end{asy}
\end{center}
\caption{$S$ with $m=3$ and $n=5$}
\end{figure}
\end{example*}
\begin{proof}
	Let $A = (0,0)$, $B = (1,n)$, and $C = (x,y)$, for integral $n,x,y$.
	
	Let $D$ be the point on $AB$ such that $CD$ is parallel to the $x$-axis.

	Let $[XYZ]$ denote the area of $\bigtriangleup XYZ$.

	Suppose $k = mi - j$, where $i,j \in \mathbb{Z}$, $i,j \geq 0$, and $j < m$. 

\begin{figure}[H]
\begin{center}
\begin{asy}
	unitsize(0.75cm);
	draw((-1,0)--(5,0),Arrow);
	draw((0,-1)--(0,5),Arrow);
	real h = 4.5;
	real x = 4;
	real y = 2;
	draw((0,0)--(0.5,h)--(x,y)--cycle);
	dot((0,0));
	label("$A$",(0,0),SW);
	dot((0.5,h));
	label("$B$",(0.5,h),NE);
	dot((x,y));
	label("$C$",(x,y),NE);
	dot((0.5*y/h,y));
	draw((0.5*y/h,y) -- (x,y));
	label("$D$",(0.5*y/h,y),SE);
\end{asy}
\end{center}
\caption{Triangle $ABC$}
\end{figure}


Observe that $[ABC] = [ACD] + [BCD] = \frac12CD \cdot y + \frac12CD\cdot(n-y) = \frac12CD\cdot n$.

Since the slope of $AB$ is $n$, $CD = x - \frac{y}n$.

So $[ABC] = \frac12n\left(x - \frac{y}n\right) = \frac12\left(xn - y\right)$.

Let $\ceil*{x}$ denote the least integer greater than $x$.

Since the only thing we're bounded by is that $x,y \in \mathbb{Z}$ such that
$0 \leq x \leq m$ and $0 \leq y \leq n$, 
let $x = \ceil*{\frac{k}{n}}$ and $y = xn - k$.
Then $[ABC] = \frac12\left(xn - y\right) = k$, 
and we know that $x \leq m$ and $y \leq n$ 
since $x > m \implies k > mn$ 
and $y > n \implies \ceil*{\frac{k}n} = x-1$, both of which are contradictions, 
so such $x,y$ exist, and we are done.

\end{proof}

\begin{example*}
	[USAMTS 2/1/33]
	Find, with proof, the minimum number positive integer $n$ with the 
	following property: for any coloring of the integers $\{1,2\dots,n\}$
	using the colors red and blue (that is, assigning the color
	"red" or "blue" to each integer in the set), there exist distinct
	integers $a,b,c$ between 1 and $n$, inclusive, all of the same color,
	such that $2a+b=c$.
\end{example*}

\begin{proof}
	Let $S$ be a coloring of $\{1,2\dots n\}$ with the required property.
Let $n$ be the size of $S$. We claim that $n=14$ is maximal and achievable.

First, we have the following lemma:
\begin{lemma*}
	If $a$ and $b$ are the same color ($a<b$, $a,b \in S$), 
	then $2a+b$, $a+2b$ and $b - 2a$ are the opposite color.
\end{lemma*}
The proof of the lemma is trivial, but it is very useful.

Without loss of generality, let 1 be colored red.
Then we will break our problem into the following cases:

\proofcase\label{red} 2 is red.
\proofcase\label{blue21} 2 is blue, 3 is red
\proofcase\label{blue22} 2 is blue, 3 is blue
\bigskip

We claim that:

In case \ref{red} $n \leq 14$, and $n=14 is achievable$.

In case \ref{blue21} $n \leq 10$

In case \ref{blue22} $n \leq 8$

\textbf{Case \ref{red}}

In this case, 4 and 5 are both blue (for $a=1, b=2$)
so 13 and 14 are both red ($(a,b) = (4,5)$)
so 11 and 12 are both blue ($(1,13)$ and $(1,14)$).
However, because $3 + 2(4) = 11$, 3 must be red. 
This means that 7 and 8 are both blue ($(1,3)$ and $(2,3)$)
We will now take a moment and look at our new list of numbers:
\begin{center}
	\red 1 2 3 \blue 4 5 \black 6 \blue 
	7 8 \black 9 10 \blue 11 12 \red 13 14
\end{center}
Now, $2(4) + 7 = 15$ must be red, but that is impossible since
$2(1) + 13$ must be blue, so $n \leq 14$.
$14 - 2(2)$ and $13 - 2(2)$ must also both be blue, and 6 can be whatever,
so here is our final set:
\begin{center}
	\red 1 2 3 \blue 4 5 \color{violet} 6 \blue 
	7 8 9 10 11 12 \red 13 14
\end{center}

\textbf{Case \ref{blue21}}

In this case, we know that 5 and 7 are both blue, so 9 and 11 are both red.
However, $2(1) + 9 = 11$ so $n \leq 10$ in case \ref{blue21}.
\begin{center}
	\red 1 \blue 2 \red 3 \black 4 \blue 5 \black 6 \blue 7 \black 8 
	\red 9 \black 10
\end{center}

We do not need to show that $n=10$ is achievable since by case \ref{red} $n=14$ is achievable.

\textbf{Case \ref{blue22}}

In this case, we know that both 7 and 8 are red, so 5 and 6 are blue
(by subtracting by 2(1)) as are 9 and 10 (by adding 2(1)).
\begin{center}
	\red 1 \blue 2 3 \black 4 \blue 5 6 \red 7 8 \blue 9 10
\end{center}
However, this is impossible, as 9 must be red since 2 and 5 are both blue.
So $n \leq 8$ in case \ref{blue22}.

Once again, we don't actually need to show that $n=8$ is achievable
since $n=14$ is achievable.

In every case $n \leq 14$, and since $n=14$ is achievable our answer is \boxed{16} and we are done.

\end{proof}

The International Baccalaureate Organization (IBO) has a number of 
assessment criteria identified for the internal assessment in mathematics,
with each criterion having level descriptors describing specific achievement
levels \cite[p. 80]{ibo_aa-guide_2021}. For the criterion on mathematical 
communication, it is very important for students to understand that 
appropriate notation, symbols, and terminology must be used in their 
written work.

As Dr. Annalisa Crannell explains in her paper “Writing in Mathematics”, 
all variables used should be defined with words describing 
what the variables represent. In particular, 
``variables in text are italicized to tell them apart from regular letters''
\cite[p. 5]{crannell_writing_1994}

\printbibliography
\end{document}
