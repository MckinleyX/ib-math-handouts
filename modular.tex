\documentclass[a4paper]{scrartcl}
\usepackage[sexy]{evan}

\title{Modular Arithmetic}
\author{Mckinley Xie}

\begin{document}
\maketitle
\begin{abstract}
	Modular arithmetic was recently removed from the IB curriculum for some reason. That makes me sad. So I'm just going to teach it here! 
\end{abstract}

\section{Introduction}
Modular arithmetic is a system of arithmetic when we consider all numbers after they're divided by some fixed number (known as the \emph{modulus}).

An example of a modular system is a clock; 5 hours from 9:00 is 2:00, because 2 is the remainder of $9 + 5 = 14$ when divided by 12.

We'll take a moment to introduce the notation:
\begin{theorem}
	We define $a \equiv b \pmod{n}$ (read ``$a$ is congrent to $b$ mod $n$'') for integers $a,b$ and positive integer $n$ if $a + kn = b$ for some integer $k \in \mathbb{Z}$.
	If $a \equiv b \pmod n$ and $c \equiv d \pmod n$, where $a,b,c,d \in \mathbb{Z}$, then:
	\begin{enumerate}
		\item $a + c \equiv b + d \pmod n$
		\item $ac \equiv bd \pmod n$
	\end{enumerate}
	(Note that, $\frac ac$ is not necessarily congruent to $\frac bd$.)
\end{theorem}
For example, $2 \equiv 12 \equiv 22 \equiv \cdots \pmod{10}$, and $7 \equiv 18 \equiv 29 \equiv \cdots \pmod{11}$.

\begin{exercise}
	Find $25 \mod 6$.
\end{exercise}
\begin{exercise}
	Find the first 3 positive integers congruent to $5$ modulo 7.
\end{exercise}


\section{Sources (and helpful links)}
\url{https://brilliant.org/wiki/modular-arithmetic/}

\end{document}
