\documentclass{scrartcl}
\usepackage[sexy]{evan}

\usepackage{asymptote} % For drawing the diagram

\usepackage{biblatex} % For bibliography
\addbibresource{ref.bib} % Use ref.bib for the bibliography file

\author{Mckinley Xie}
\title{Unit 2 \LaTeX{} Project}

\begin{document}
\maketitle
\begin{example*}
	[Komal Problem B. 5188]
	Prove that the height of a circumscribed trapezium 
	cannot be greater than the geometric mean of the bases. 
\end{example*}
\begin{proof}
	Consider the left-hand side of the following diagram:
	\bigskip
	\begin{center}
\begin{asy}
	import olympiad;
	unitsize(2cm);
	draw(unitcircle);
	pair a = (-0.8,1);
	pair c = (1.5,-1);
	pair p = tangent(a,origin,1,2);
	pair q = tangent(c,origin,1,1);
	pair r = tangent(c,origin,1,2);
	pair s = tangent(a,origin,1,1);
	pair b = extension(a,p,c,q);
	pair d = extension(a,s,c,r);
	dot(a);
	dot(b);
	dot(c);
	dot(d);
	dot(p);
	dot(q);
	dot(r);
	dot(s);
	dot(origin);
	label("$A$",a,NW);
	label("$B$",b,NE);
	label("$C$",c,SE);
	label("$D$",d,SW);
	label("$P$",p,N);
	label("$Q$",q,E);
	label("$R$",r,S);
	label("$S$",s,W);
	label("$O$",origin,E);
	label("$\omega$",(0.75,-0.7),E);
	draw(a--b--c--d--cycle);
	draw(p--r);
	draw(a--origin);
	draw(d--origin);
	draw(s--origin);
\end{asy}
	\end{center}
	\newcommand{\tri}{\triangle}

Observe that $AP=AS$ and $PO=OS$, and $\angle OSA = \angle OPA = 90 ^\circ$
\cite{mw_cirtan}, so $\tri APO \cong \tri ASO$.
Similarly, $\tri ORD \cong \tri ODS$.

Because $AO$ bisects $\angle POS$ and $DO$ bisects $\angle SOR$,
$\angle AOD = \frac{180^\circ}{2} = 90^\circ$.
Because $\angle SAO = \angle OAS$, $\tri AOD \sim \tri ASO$.
Similarly, $\tri OSD \sim \tri AOD$.

Let $r$ be the radius of circle $\omega$ and $c = \frac{SO}{AS}$. 
Then $AS = \frac{r}{c}$, and by similar trinagles $SD = rc$.

Repeating this argument for the right-hand side we have 
$BQ = \frac{r}{d}$, $OQ = r$, and $OC = rd$ for some constant $d$.

Let $g$ be the geometric mean of the bases.
\begin{align*}
	g &= \sqrt{\left(\frac{r}{c} + \frac{r}{d}\right)\left(rc + rd\right)} \\
	&= r\sqrt{\left(\frac1c + \frac1d\right)\left(c+d\right)} \\
	&= r\sqrt{1 + \frac{c}d + \frac{d}c + 1}
\end{align*}
Finally, by AM-GM\cite{mw_amgm} on $\frac{c}d + \frac{d}c$,
\[ g \geq r\sqrt{1 + 2 + 1} = 2r\]
And since $2r$ is equal to the height of the trapezium, we are done.
\end{proof}
\nocite{mw_simtri} 
\printbibliography
\end{document}
