\documentclass[11pt]{scrartcl}
\usepackage[retro,cabin,sleek]{dylanadi}

\title{dylanadi.sty}
\author{Dylan Yu}
\date{\today}
\changemaincolor{Emerald}
\changesecondcolor{Periwinkle}

\begin{document}

\dylantitle
\tableofcontents

\section{Theorems}
\begin{notation}
\item For a triangle $ABC$, $AB=c,BC=a,CA=b$.
\item The semiperimeter of $\triangle ABC$ is denoted as $s=\frac{a+b+c}{2}$.
\end{notation}
\begin{theorem}[Pythagorean Theorem]
For positive $a,b,c$, 
$$a^2+b^2=c^2.$$
\end{theorem}
\begin{proof}
We will apply the following claim:
\begin{claim}
All squares are rectangles.
\end{claim}
Just kidding.
\end{proof}
\begin{example}
What is $3^2+4^2$?
\end{example}
\begin{walk}~
\begin{enumerate}
    \item Where have we seen some of squares?
    \item That's right -- Pythagorean Theorem!
    \item Thus, we can take advantage of a 3-4-5 triangle.
    \item Alternatively, \textbf{just calculate it}.
\end{enumerate}
\end{walk}
\begin{defn}[Theorem]
A \vocab{theorem} is a general proposition proved by a chain of reasoning.
\end{defn}
\begin{fact}
This is important!
\end{fact}
\begin{remark}
This is also important.
\end{remark}
\begin{boxpar}[Paragraph]
Nice looking box.
\end{boxpar}
\begin{caution}
This is a warning box, in case there's something important to talk about. For example, a bogus solution could go here. It helps denote what could possibly be incorrect: i.e.,
$$a^3+b^3=c^3$$
in a right triangle is wrong (if they represent the side lengths of the triangle).
\end{caution}
\begin{exercisebox}
\begin{exercise}[AIME II 2020/3]
The value of $x$ that satisfies $\log_{2^x} 3^{20} = \log_{2^{x+3}} 3^{2020}$ can be written as $\frac{m}{n}$, where $m$ and $n$ are relatively prime positive integers. Find $m+n$.
\end{exercise}
\begin{exercise}[AIME 1986/8]
Let $S$ be the sum of the base $10$ logarithms of all the proper divisors (all divisors of a number excluding itself) of $1000000$. What is the integer nearest to $S$?
\end{exercise}
\begin{exercise}[AIME I 2020/2]
There is a unique positive real number $x$ such that the three numbers $\log_8{2x}$, $\log_4{x}$, and $\log_2{x}$, in that order, form a geometric progression with positive common ratio. The number $x$ can be written as $\frac{m}{n}$, where $m$ and $n$ are relatively prime positive integers. Find $m + n$.
\end{exercise}
\end{exercisebox}
\begin{question}
This is a question. What is your question?
\end{question}

\problems
\minpt{32}

\psetquote{I like cheeseburgers.}{Anonymous} 

\prob[2]{AIME I 2007/7}{Let $N = \sum\limits_{k = 1}^{1000} k ( \lceil \log_{\sqrt{2}} k \rceil  - \lfloor \log_{\sqrt{2}} k \rfloor ).$ Find the remainder when $N$ is divided by 1000. ($\lfloor{k}\rfloor$ is the greatest integer less than or equal to $k$, and $\lceil{k}\rceil$ is the least integer greater than or equal to $k$.)}

\req[3]{AIME II 2010/5}{Positive numbers $x$, $y$, and $z$ satisfy $xyz = 10^{81}$ and $(\log_{10}x)(\log_{10} yz) + (\log_{10}y) (\log_{10}z) = 468$. Find $\sqrt {(\log_{10}x)^2 + (\log_{10}y)^2 + (\log_{10}z)^2}$.}

\begin{hardbox}
\prob[9]{AIME I 2012/9}{Let $x,$ $y,$ and $z$ be positive real numbers that satisfy \[2\log_{x}(2y) = 2\log_{2x}(4z) = \log_{2x^4}(8yz) \ne 0.\] The value of $xy^5z$ can be expressed in the form $\frac{1}{2^{p/q}},$ where $p$ and $q$ are relatively prime positive integers. Find $p+q.$}
\end{hardbox}

\problem{Problem without points or source.}

\begin{Problem}[Hello]
It's me.
\end{Problem}

\end{document}
